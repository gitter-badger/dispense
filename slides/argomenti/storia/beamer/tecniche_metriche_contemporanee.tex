%
% $Id: tecniche_metriche_contemporanee.tex 8 2014-02-04 21:01:21Z nicb $
%
\svnInfo $Id: tecniche_metriche_contemporanee.tex 8 2014-02-04 21:01:21Z nicb $

\setcounter{ms}{0}
\refstepcounter{ms}
\begin{frame}
    \frametitle{Tecniche metriche contemporanee (\arabic{ms})}

    \begin{itemize}

        \item Metri variabili

        \item Metri asimmetrici

        \item Metri tradizionali con disposizioni metriche asimmetriche

    \end{itemize}

\end{frame}

\refstepcounter{ms}
\begin{frame}
    \frametitle{Tecniche metriche contemporanee (\arabic{ms})}

    \begin{center}
        \begin{figure}
            \pgfimage[height=0.7\textheight]{\imagedir/babbitt-piano}
            \caption{Milton Babbit, \emph{Three Compositions for Piano} (1947), n.1, batt.9--12}
        \end{figure}
    \end{center}

\end{frame}

\refstepcounter{ms}
\begin{frame}
    \frametitle{Tecniche metriche contemporanee (\arabic{ms})}

    dove :

    \begin{tabular}{l *{4}{c}}
             & batt.9 & batt.10 & batt.11 & batt.12\\
        MD:  & pausa  & 6+6     & 6+6     & 1+5+2+4\\
        MS:  & 5+1+4+2 & 4+2+5+1 & 6+6     & pausa\\
    \end{tabular}

\end{frame}

\newcounter{dp}
\setcounter{dp}{0}
\refstepcounter{dp}
\refstepcounter{ms}
\begin{frame}
    \frametitle{Tecniche metriche contemporanee (\arabic{ms})}

    {\scriptsize
    \begin{itemize}

        \item Dispositivi polimetrici (\arabic{dp}):

            \begin{enumerate}[A) ]
            {\scriptsize

                \item stesso metro sfasato

                    \begin{center}
                        \pgfimage[height=0.15\textheight]{\imagedir/polimetri-A}
                    \end{center}

                \item\label{B_meters} metri diversi con battute coincidenti

                    \begin{center}
                        \pgfimage[height=0.15\textheight]{\imagedir/polimetri-B}
                    \end{center}
    
                \item metri diversi e battute non coincidenti

                    \begin{center}
                        \pgfimage[height=0.15\textheight]{\imagedir/polimetri-C}
                    \end{center}

            }
            \end{enumerate}

    \end{itemize}
    }

\end{frame}

\refstepcounter{ms}
\begin{frame}
    \frametitle{Tecniche metriche contemporanee (\arabic{ms})}

    Esempio di polimetro di tipo \ref{B_meters} (raro):

    \begin{center}
        \begin{figure}
            \pgfimage[height=0.65\textheight]{\imagedir/stravinskij-petrouchka}
            \caption{Igor Stravinskij, \emph{Petru\v{s}ka} (1911), primo quadro}
        \end{figure}
    \end{center}

\end{frame}

\refstepcounter{ms}
\refstepcounter{dp}
\begin{frame}
    \frametitle{Tecniche metriche contemporanee (\arabic{ms})}

    \begin{itemize}

        \item Dispositivi polimetrici (\arabic{dp}):

        \begin{itemize}

            \item polimetri nascosti:

                \begin{center}
                    \begin{figure}
                        \pgfimage[height=0.6\textheight]{\imagedir/bartok-quartet-III}
                        \caption{Bela Bartok, \emph{Quartetto n.3} (1927), II}
                    \end{figure}
                \end{center}

        \end{itemize}

    \end{itemize}

\end{frame}

\refstepcounter{ms}
\refstepcounter{dp}
\begin{frame}
    \frametitle{Tecniche metriche contemporanee (\arabic{ms})}


    che in notazione esplicita diventano:

    \begin{center}
        \pgfimage[width=0.9\textwidth]{\imagedir/bartok-quartet-III-riscritto}
    \end{center}

\end{frame}
