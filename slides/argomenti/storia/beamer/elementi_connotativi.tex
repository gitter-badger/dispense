%
% $Id: elementi_connotativi.tex 8 2014-02-04 21:01:21Z nicb $
%
\svnInfo $Id: elementi_connotativi.tex 8 2014-02-04 21:01:21Z nicb $

\setcounter{ms}{0}
\refstepcounter{ms}
\begin{frame}
    \frametitle{Utilizzazione di elementi connotativi (\arabic{ms})}

    \begin{itemize}

        \item La riproducibilit\`a dei suoni ha messo in luce
            le valenze connotative di questi ultimi, rimaste inesplorate

        \item i compositori contemporanei hanno colto ed elaborato questi elementi
            in modi diversi, quali:

            \begin{itemize}

                \item neo--naturalismo

                \item sovrapposizioni stilistiche

                \item sovrapposizioni letterali

                \item tematismo connotativo
            \end{itemize}

    \end{itemize}

\end{frame}

\refstepcounter{ms}
\begin{frame}
    \frametitle{Utilizzazione di elementi connotativi (\arabic{ms})}

    \begin{itemize}

        \item neo--naturalismo
            (esempio sonoro: Charles Ives, \emph{Central Park in the Dark} (1906), per orchestra sinfonica)

        \item sovrapposizioni stilistiche
           (esempio sonoro: Alfred \v{S}nitke,
        \emph{Concerto Grosso n.1}, I$\degree$ mov. (1976-77),
        per due violini, clavicembalo, pianoforte preparato
        e orchestra d'archi)
%
        \item sovrapposizioni letterali (esempio sonoro:
            Luciano Berio, \emph{Sinfonia}, III$\degree$ mov. (1968),
            per ottetto vocale amplificato e orchestra sinfonica)

        \item tematismo connotativo (esempio sonoro:
            Nicola Bernardini, \emph{Partita per voce sola} (1985),
                per voce femminile)
%
    \end{itemize}

\end{frame}
