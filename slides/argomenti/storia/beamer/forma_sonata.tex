%
% $Id: forma_sonata.tex 8 2014-02-04 21:01:21Z nicb $
%
\svnInfo $Id: forma_sonata.tex 8 2014-02-04 21:01:21Z nicb $

\setcounter{ms}{0}
\refstepcounter{ms}
\begin{frame}
    \frametitle{Form(e) Sonata (\arabic{ms})}

    \begin{itemize}

        \item La forma sonata, la cui espressivit\`a risiede
              in gran parte nell'esposizione e nella risoluzione
              di conflitti fra tonalit\`a, perde la sua forza
              in un contesto tonale indebolito o polverizzato.

        \item Ci\`o nonostante, molti autori utilizzano
              la forma sonata come canovaccio
              re-interpretandone i principi.

    \end{itemize}

\end{frame}

\refstepcounter{ms}
\begin{frame}
    \frametitle{Form(e) Sonata (\arabic{ms})}

    \begin{itemize}

        \item Esempio: Bela Bartok, \emph{Quartetto n.6} primo mov. (1939)

        \item {\tiny \begin{tabular}{*{4}{p{0.15\textwidth}}}
                    Esposizione & Primo Tema & batt.24 & Re maggiore\\
                    \hhline{~---}
                                & Secondo Tema & batt.81$\Rightarrow$94 & Do maggiore\\
                                &  &  & $\Rightarrow$\\
                                &  &  & Fa maggiore\\
                    \hhline{~---}
                               & Terzo Tema & batt.99$\Rightarrow$110$\Rightarrow$157 & Mi bemolle\\
                               &  &  & $\Rightarrow$\\
                               &  &  & Fa maggiore\\
                    \hhline{====} 
                       Sviluppo &  & batt.158$\Rightarrow$267 &\\
                    \hhline{====} 
                        Ricapitolazione & Primo Tema & batt.268 & Re maggiore\\
                    \hhline{~---}
                               & Secondo Tema & batt.312 & Do diesis maggiore\\
                    \hhline{~---}
                                 & Terzo Tema & batt.343$\Rightarrow$352 & Si bemolle\\
                                 &  &  & $\Rightarrow$\\
                                 &  &  & Re maggiore\\
                \end{tabular}}

    \end{itemize}

\end{frame}
