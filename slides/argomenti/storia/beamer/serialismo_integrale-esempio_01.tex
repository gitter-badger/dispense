%
% $Id: serialismo_integrale-esempio_01.tex 8 2014-02-04 21:01:21Z nicb $
%
\svnInfo $Id: serialismo_integrale-esempio_01.tex 8 2014-02-04 21:01:21Z nicb $

\setcounter{ms}{0}
\refstepcounter{ms}
\begin{frame}
    \frametitle{Serialismo integrale - Esempio 1 (\arabic{ms})}

    \begin{center}
        \begin{figure}
            \pgfimage[height=0.7\textheight]{\imagedir/boulez-structures-1a-1}
            \caption{Pierre Boulez, \emph{Structures Ia}, (1952) batt.1--7}
        \end{figure}
    \end{center}

\end{frame}

\refstepcounter{ms}
\begin{frame}
    \frametitle{Serialismo integrale - Esempio 1 (\arabic{ms})}

    \begin{itemize}

        \item Organizzazione del pezzo:

        \begin{itemize}

            \item La serie delle altezze \`e:

                \vspace{\baselineskip}
                \pgfimage[width=0.85\textwidth]{\imagedir/boulez-structures-1a-row}

                \vspace{\baselineskip}
                (basata sulla serie di
                \emph{Modes de valeurs et d'intensit\'es}
                di Olivier Messiaen (1949))

        \end{itemize}

    \end{itemize}

\end{frame}

\refstepcounter{ms}
\begin{frame}
    \frametitle{Serialismo integrale - Esempio 1 (\arabic{ms})}

    \begin{itemize}

            \item Viene costruita una matrice di trasposizioni basata
                  sul numero d'ordine progressivo delle note:

                \pgfimage[width=0.9\textwidth]{\imagedir/boulez-structures-1a-row-devel}

            \item una matrice simile viene costruita per la serie inversa

    \end{itemize}

\end{frame}

\refstepcounter{ms}
\begin{frame}
    \frametitle{Serialismo integrale - Esempio 1 (\arabic{ms})}

    \begin{itemize}

        \item \emph{Structures Ia} consta di 576 note,
                cio\`e 12 note per 48 forme seriali

        \item il pianoforte I suona tutte le trasposizioni
            dell'originale e dell'inversione retrograda

        \item il pianoforte II suona tutte le trasposizioni
            dell'inversione e dell'originale retrogrado

        \item le durate delle note dipendono dalle due matrici,
              ed il numero d'ordine delle note indica la durata
                in semibiscrome:

            \begin{itemize}

                \item il pianoforte I inizia con il retrogrado
                      dell'ultima trasposizione delle inversioni

                \item il pianoforte II inizia con il retrogrado
                      dell'ultima trasposizione dell'originale

            \end{itemize}

    \end{itemize}

\end{frame}

\refstepcounter{ms}
\begin{frame}
    \frametitle{Serialismo integrale - Esempio 1 (\arabic{ms})}

    \begin{itemize}
        \item le dinamiche sono serializzate attraverso
            dodici livelli (da \emph{pppp} a \emph{ffff})
            che dipendono da uno scorrimento diagonale
            delle matrici

        \item l'articolazione \`e serializzata attraverso
              10 ``modalit\`a'', sempre dipendenti da
              uno scorrimento diagonale delle matrici

    \end{itemize}

\end{frame}

\refstepcounter{ms}
\begin{frame}
    \frametitle{Serialismo integrale - Esempio 1 (\arabic{ms})}

    \begin{itemize}

        \item Parametri non serializzati:

        \begin{itemize}

            \item timbro

            \item registro (l'unica costrizione \`e che quando una nota
                appare simultaneamente nei due pianoforti
                essa viene suonata nello stesso registro)

            \item metro

        \end{itemize}

    \end{itemize}

\end{frame}
