%
% $Id: canoni_fughe.tex 8 2014-02-04 21:01:21Z nicb $
%
\svnInfo $Id: canoni_fughe.tex 8 2014-02-04 21:01:21Z nicb $

\setcounter{ms}{0}
\refstepcounter{ms}
\begin{frame}
    \frametitle{Canoni e fughe (\arabic{ms})}

    \begin{itemize}

        \item forme contrappuntistiche che si adattano bene alle necessit\`a
              espressive della musica del novecento.

        \item Esempi:

            \begin{itemize}

                \item Arnold Sch\"onberg, \emph{F\"unf Orchesterst\"ucke -- Farben} op.16 n.3 (1909)

                \item Arnold Sch\"onberg, \emph{Suite}, op.25 (1923), trio del minuetto

                \item Anton Webern, \emph{F\"unf Kanons}, op.16 (1924)

                \item Terry Riley, \emph{In C} (1964)
                      canone formalizzato come segue:
                      53 frasi musicali additive suonate da
                      ciascuno strumentista partendo da frasi
                      diverse e facendo percorsi diversi

            \end{itemize}

    \end{itemize}

\end{frame}

\refstepcounter{ms}
\begin{frame}
    \frametitle{Canoni e fughe (\arabic{ms})}

    \begin{itemize}

        \item Esempi:

            \begin{itemize}

                \item Paul Hindemith, \emph{Ludus Tonalis}, (1942),
                    dodici fughe separate da undici interludi, nelle tonalit\`a che seguono:

                    {\scriptsize \begin{tabular}{*{12}{p{0.01\textwidth}}}
                        Do & Sol & Fa & La & Mi & Mib &
                        Lab & Re & Sib & Reb & Si & Fa\#\\
                    \end{tabular}}

                \item Bela Bartok, \emph{Musica per Strumenti ad Arco, Percussioni e Celesta},
                      op. 36 primo movimento (1936)
                    
            \end{itemize}

    \end{itemize}

\end{frame}

