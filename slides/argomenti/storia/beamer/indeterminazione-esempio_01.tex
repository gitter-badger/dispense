%
% $Id: indeterminazione-esempio_01.tex 8 2014-02-04 21:01:21Z nicb $
%
\svnInfo $Id: indeterminazione-esempio_01.tex 8 2014-02-04 21:01:21Z nicb $

\setcounter{ms}{0}
\refstepcounter{ms}
\begin{frame}
    \frametitle{Indeterminazione come principio costruttivo - Un esempio (\arabic{ms})}

    \begin{center}
        \begin{figure}
            \pgfimage[height=0.6\textheight]{\imagedir/maderna-satellite}
            \caption{Bruno Maderna, \emph{Serenata per un satellite}, (1969)}
        \end{figure}
    \end{center}

\end{frame}

\refstepcounter{ms}
\begin{frame}
    \frametitle{Indeterminazione come principio costruttivo - Un esempio (\arabic{ms})}

    \begin{itemize}

        \item Le indicazioni indicate nella partitura di \emph{Serenata per un Satellite} sono:

        \item ``possono suonarla: violino, flauto (anche ottavino),
                oboe (anche oboe d'amore - anche musette),
                clarinetto (trasponendo naturalmente la parte),
                marimba, arpa, chitarra e mandolino (suonando quello che possono!) --
                tutti insieme o separati o a gruppi --
                improvvisando insomma MA! --
                con le note scritte.''

        \item ``durata: da un minimo di 4' a 12' ''

    \end{itemize}

\end{frame}

\refstepcounter{ms}
\begin{frame}
    \frametitle{Indeterminazione come principio costruttivo - Un esempio (\arabic{ms})}

    \begin{itemize}

        \item Rimangono quindi indeterminati:

        \begin{itemize}

            \item la forma del brano

            \item l'organico

            \item i frammenti suonati da ciascuno strumento
                (anche se la scrittura indica che il compositore
                ha scritto determinati frammenti
                con un strumento particolare in mente)

        \end{itemize}

    \end{itemize}

\end{frame}
