%
% $Id: movimenti_paralleli.tex 8 2014-02-04 21:01:21Z nicb $
%
\svnInfo $Id: movimenti_paralleli.tex 8 2014-02-04 21:01:21Z nicb $

\setcounter{ms}{0}
\refstepcounter{ms}
\begin{frame}
    \frametitle{Movimenti paralleli (\arabic{ms})}

    \begin{center}
        \begin{figure}
            \pgfimage[width=0.9\textwidth]{\imagedir/gotterdammerung}
            \caption{\listento{run: ogg123 "\exampledir/Wagner-Gotterdammerung-Dritter\_Aufz-Szene\_3-Was\_hor\_ich? (fragment).ogg"}{Wagner, \emph{G\"otterd\"ammerung} (1874), Atto III, Scena 2}}
        \end{figure}
    \end{center}

\end{frame}

\refstepcounter{ms}
\begin{frame}
    \frametitle{Movimenti paralleli (\arabic{ms})}

    Per rappresentare un volo di uccelli, Wagner utilizza:

    \begin{itemize}
        \item una successione non funzionale di accordi di settima semi-diminuita
        \item in questo caso, le voci perseguono ciascuna un proprio obbiettivo separato
        \item muovendosi cromaticamente o per toni vicini
    \end{itemize}

\end{frame}

