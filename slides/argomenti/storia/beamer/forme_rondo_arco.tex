%
% $Id: forme_rondo_arco.tex 8 2014-02-04 21:01:21Z nicb $
%
\svnInfo $Id: forme_rondo_arco.tex 8 2014-02-04 21:01:21Z nicb $

\setcounter{ms}{0}
\refstepcounter{ms}
\begin{frame}
    \frametitle{Forme rond\`o e forme ad arco (\arabic{ms})}

    \begin{itemize}

        \item Tradizionalmente

                \begin{tabular}{*{6}{c} p{0.1\textwidth} p{0.25\textwidth}}
                     & A & B & A & C & A &  & (cinque parti)\\
                     \multicolumn{8}{l}{oppure}\\
                     & A & B & A & B & A &  & (cinque parti)\\
                     \multicolumn{8}{l}{oppure}\\
                     A & B & A & C & A & B & A & (sette parti)\\
                \end{tabular}

        \item normalmente le parti sono a contrasto
            (armonico, timbrico, di tessitura, ecc.)

    \end{itemize}

\end{frame}

\refstepcounter{ms}
\begin{frame}
    \frametitle{Forme rond\`o e forme ad arco (\arabic{ms})}

    \begin{itemize}

        \item utilizzate in forma non convenzionale nella musica del '900,
            e in particolare:

        \begin{itemize}

            \item sequenze di tonalit\`a non convenzionali

            \item riduzioni a forme ad arco con sequenze non convenzionali

            \item ecc.

        \end{itemize}

    \end{itemize}

\end{frame}
