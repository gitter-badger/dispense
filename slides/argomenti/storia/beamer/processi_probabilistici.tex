%
% $Id: processi_probabilistici.tex 8 2014-02-04 21:01:21Z nicb $
%
\svnInfo $Id: processi_probabilistici.tex 8 2014-02-04 21:01:21Z nicb $

\setcounter{ms}{0}
\refstepcounter{ms}
\begin{frame}
    \frametitle{Processi statistici e probabilistici (\arabic{ms})}

    \begin{itemize}

        \item L'introduzione di principi statistici e
            probabilistici nell'elaborazione
            di processi musicali \`e riconducibile
            all'opera del compositore greco
            naturalizzato francese Iannis Xenakis.

    \end{itemize}

\end{frame}

\refstepcounter{ms}
\begin{frame}
    \frametitle{Processi statistici e probabilistici (\arabic{ms})}

    \begin{itemize}
        \item Tra le tecniche usate
              e descritte da Xenakis nelle sue composizioni:

            \begin{itemize}

                \item distribuzioni probabilistiche di elementi
                    (ad es. distribuzioni gaussiane di frequenze, del loro movimento, ecc.)

                \item matematica insiemistica per regolare insiemi discreti
                    (ad es. spazi temperati, insiemi strumentali, ecc.)

                \item processi stocastici
                    (catene di Markov, moto browniano, ecc.)
                    per la concatenazione di eventi

            \end{itemize}

    \end{itemize}

\end{frame}
