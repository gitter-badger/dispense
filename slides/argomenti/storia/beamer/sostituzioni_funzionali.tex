%
% $Id: sostituzioni_funzionali.tex 8 2014-02-04 21:01:21Z nicb $
%
\svnInfo $Id: sostituzioni_funzionali.tex 8 2014-02-04 21:01:21Z nicb $

\setcounter{ms}{0}
\refstepcounter{ms}
\begin{frame}
    \frametitle{Sostituzioni funzionali (\arabic{ms})}
        
        \begin{itemize}

            \item armonia classica -- note funzionali: tonica, mediante, settima

            \item sostituzioni classiche (accordi con note funzionali in comune):

            \begin{itemize}

                \item funzione tonica: accordi sul primo grado,
                    accordi sul sesto grado, accordi sul terzo grado

                \item funzione sottodominante: accordi sul quarto grado,
                    accordi sul secondo grado

                \item funzione dominante: accordi sul quinto grado, accordi sul settimo grado

            \end{itemize}

        \end{itemize}

\end{frame}
            
\refstepcounter{ms}
\begin{frame}
    \frametitle{Sostituzioni funzionali (\arabic{ms})}
        
        \begin{itemize}

            \item la sostituzione funzionale permette l'estensione della tavolozza
                di colori armonici senza variare la funzionalit\`a tonale degli accordi

            \item Il meccanismo si estende attraverso le dominanti secondarie,
                le dominanti estese, ecc.

        \end{itemize}

\end{frame}

\refstepcounter{ms}
\begin{frame}
  \frametitle{Sostituzioni funzionali (\arabic{ms})}
        
    \begin{itemize}

        \item All'inizio del '900 il meccanismo delle sostituzioni \`e stato ormai
            sfruttato in tutto l'insieme cromatico

        \item Sch\"onberg chiama le funzioni ``regioni tonali'' riassumendo cos\`i le
            relazioni tra la tonica e gli altri accordi (in base alla quantit\`a di note in comune):

        \begin{itemize}

            \item dirette
            \item indirette vicine
            \item indirette
            \item indirette remote
            \item distanti

        \end{itemize}

    \end{itemize}

\end{frame}

\refstepcounter{ms}
\begin{frame}
  \frametitle{Sostituzioni funzionali (\arabic{ms})}
        
	% quadro delle sostituzioni di funzione di Sch\(:onberg
	% regioni maggiori
	\begin{center} 
	    \begin{figure}
          \pgfimage[height=0.6\textheight]{\imagedir/major-regions}
	        \caption{Regioni delle tonalit\`a maggiori
	            (cf. Arnold Sch\"onberg, \emph{The Structural Functions of Harmony},
	            W.W.Norton \& Company, 1954, p.20)}
	    \end{figure}
	\end{center} 

\end{frame}

\refstepcounter{ms}
\begin{frame}
  \frametitle{Sostituzioni funzionali (\arabic{ms})}
        
	% quadro delle sostituzioni di funzione di Sch\(:onberg
	% regioni minori
	\begin{center} 
	    \begin{figure}
	        \pgfimage[height=0.6\textheight]{\imagedir/minor-regions}
	        \caption{Regioni delle tonalit\`a minori
	            (cf. Arnold Sch\"onberg, \emph{The Structural Functions of Harmony},
	            W.W.Norton \& Company, 1954, p.30)}
	    \end{figure}
	\end{center} 

\end{frame}

