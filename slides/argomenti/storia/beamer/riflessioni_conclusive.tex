%
% $Id: riflessioni_conclusive.tex 8 2014-02-04 21:01:21Z nicb $
%
\svnInfo $Id: riflessioni_conclusive.tex 8 2014-02-04 21:01:21Z nicb $

\setcounter{ms}{0}
\refstepcounter{ms}
\begin{frame}
    \frametitle{Riflessioni sul pensiero compositivo contemporaneo}

    \begin{itemize}

        \item A conclusione di questa panoramica, \`e opportuno sottolineare che
            oggi l'evoluzione dei linguaggi compositivi

        \begin{itemize}

            \item non \`e pi\`u univoca

            \item non riguarda pi\`u un'evoluzione di tipo grammaticale:
                la ricerca tende ad espandere i suoi confini
                oltre gli aspetti puramente sintattici
                dell'espressione musicale per abbracciare il timbro,
                gli aspetti connotativi, ecc.

            \item rivolge un'estrema attenzione,
                dividendosi nettamente in contrapposizioni estetiche,
                agli aspetti articolatori del materiale e cio\`e:

                \begin{itemize}

                    \item l'articolazione del silenzio

                    \item una musica fatta di note vs.  una musica fatta di suoni

                \end{itemize}

        \end{itemize}

    \end{itemize}

\end{frame}
