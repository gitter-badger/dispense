%
% $Id: GRM.tex 8 2014-02-04 21:01:21Z nicb $
%
\svnInfo $Id: GRM.tex 8 2014-02-04 21:01:21Z nicb $

\setcounter{ms}{0}
\refstepcounter{ms}
\begin{frame}
    \frametitle{Il \emph{GRM} - (\arabic{ms})}

    \begin{itemize}

    \item Il \emph{GRM} nasce dalle prime esperienze compiute da Pierre Schaeffer,
        allora dirigente di \emph{Radio France}, nel 1948.
        Tali esperienze erano mirate all'utilizzazione di suoni concreti
        (cio\`e di suoni e rumori ambientali registrati) a fini musicali.

    \item Il \emph{GRM} nasce quindi ufficialmente nel 1951
        dalla collaborazione di Schaeffer con il compositore Pierre Henry,
        e immediatamente attira l'interesse di compositori quali
        Messiaen, Boulez, Stockhausen, Milhaud, ecc.

    \item La caratteristica principe di questo centro \`e da sempre

        \begin{itemize}

            \item l'attenzione per il suono concreto

            \item la tecnologia necessaria alla sua elaborazione

        \end{itemize}

    \item Il \emph{GRM} \`e tuttora attivo.

    \end{itemize}

\end{frame}

\refstepcounter{ms}
\begin{frame}
    \frametitle{Il \emph{GRM} - (\arabic{ms})}

    \begin{itemize}

        \item Il \emph{GRM} ha anche una forte tradizione di invenzioni tecnologiche
            che arriva sino al giorno d'oggi con i \emph{GRM-tools},
            plugins per sequencer creati da Emmanuel Favreau; all'inizio furono creati:

        \begin{itemize}

            \item magnetofoni a velocit\`a variabile

            \item il \emph{phonog\`ene} (Schaeffer e Poullin),
                magnetofono con scorrimento del nastro variabile
                secondo la scala temperata

            \item il \emph{morphophone} (Poullin e Moles),
                sistema di echi artificiali multipli

        \end{itemize}

    \end{itemize}

\end{frame}

\refstepcounter{ms}
\begin{frame}
    \frametitle{Il \emph{GRM} - (\arabic{ms})}

    \begin{center}
        \begin{figure}
            \pgfimage[height=0.7\textheight]{\imagedir/GRM-01}
            \caption{Pierre Schaeffer nel primo studio del \emph{GRM}, 1951}
        \end{figure}
    \end{center}

\end{frame}
