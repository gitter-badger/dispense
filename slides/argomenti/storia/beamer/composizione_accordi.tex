%
% $Id: composizione_accordi.tex 8 2014-02-04 21:01:21Z nicb $
%
\svnInfo $Id: composizione_accordi.tex 8 2014-02-04 21:01:21Z nicb $

\setcounter{ms}{0}
\refstepcounter{ms}
\begin{frame}
    \frametitle{Composizione degli accordi (\arabic{ms})}
    
Oltre alle alterazioni degli accordi convenzionali
utilizzate in abbondanza nella musica romantica,
si osservano:

    \begin{itemize}

        \item accordi di terze con doppie medianti, toniche, quinte e settime

        \item configurazioni diverse, quali

            \begin{itemize}

                \item accordi costruiti su seconde (settime)
                    (anche soli toni interi)

                \item accordi costruiti su quarte (quinte)

                \item accordi costruiti su intervalli misti

            \end{itemize}

        \item accordi politonali

    \end{itemize}

\end{frame}

\refstepcounter{ms}
\begin{frame}
    \frametitle{Composizione degli accordi (\arabic{ms})}
    
    Esempio: l'\emph{accordo Petru\v{s}ka} (Do/Fa\#)

    \begin{center}
    \begin{figure}
        \pgfimage[width=0.9\textwidth]{\imagedir/petruschka}
        \caption{\listento{run: play \exampledir/petruschka.wav}{Stravinksij, \emph{Petru\v{s}ka} (1911), Second Tableau}}
    \end{figure}
        \listento{run: play \exampledir/petruschka-low.wav}{(parte inferiore)}
        \listento{run: play \exampledir/petruschka-hi.wav}{(parte superiore)}
    \end{center}

\end{frame}

\refstepcounter{ms}
\begin{frame}
    \frametitle{Composizione degli accordi (\arabic{ms})}
    
	La politonalit\`a si distingue dagli accordi
	di terze con 11me/13me aggiunte, ecc.
	quando vengono utilizzati altri mezzi
	per asserirla, come:

  \begin{itemize}

      \item la separazione del movimento delle voci

      \item la separazione dei registri

      \item la separazione dei timbri

      \item la separazione delle progressioni

   \end{itemize}

\end{frame}
