%
% $Id: armonia_cromatica.tex 8 2014-02-04 21:01:21Z nicb $
%
\svnInfo $Id: armonia_cromatica.tex 8 2014-02-04 21:01:21Z nicb $

\begin{frame}
    \frametitle{Armonia Cromatica}

    Nell'armonia cromatica il meccanismo si inverte.
    Mantenendo il principio delle note comuni e/o vicine
    vengono stabilite relazioni non funzionali dal punto di vista tonale:

    \begin{itemize}

        \item relazione cromatica delle medianti
        \item sequenze per toni vicini
        \item sequenze reali (trasposizioni non-diatoniche)
        \item movimenti paralleli delle voci
        \item sistemi alternativi di composizione degli accordi
        \item sospensione della tonalit\`a

    \end{itemize}

\end{frame}

\begin{frame}
    \frametitle{Relazioni cromatiche delle medianti}

    \begin{itemize}
    {\scriptsize

        \item due accordi si trovano in relazione cromatica
            delle medianti quando sono della stessa qualit\`a (maggiore o minore)
            e le loro note fondamentali si trovano ad un intervallo di terza
            (maggiore o minore) 

        \item Esempi:

            \begin{center}
                \begin{figure}
                    \pgfimage[height=0.17\textheight]{\imagedir/chrom-mediants-maj.png}\\[0.25\baselineskip]
                    \pgfimage[height=0.17\textheight]{\imagedir/chrom-mediants-min.png}
                    \caption{\scriptsize Relazioni cromatiche delle medianti}
                \end{figure}
            \end{center}

        \item \listento{run: timidity \imagedir/chrom-mediants-maj.midi}{(maggiori)} \listento{run: timidity \imagedir/chrom-mediants-min.midi}{(minori)}
        \item Estensioni: relazioni doppiamente cromatiche (ad es. Do mib)

    }
    \end{itemize}

\end{frame}
