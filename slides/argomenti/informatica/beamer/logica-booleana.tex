%
% Author: Nicola Bernardini <nicb@sme-ccppd.org>
%
% Copyright (c) 2004 Nicola Bernardini
% Copyright (c) 2004 Conservatorio "C.Pollini", Padova
%
% This work is licensed under the Creative Commons 
% Attribution-ShareAlike License. To 
% view a copy of this license, visit 
% http://creativecommons.org/licenses/by-sa/2.0/ 
% or send a letter to Creative Commons, 
% 559 Nathan Abbott Way, Stanford, California 94305, USA.
%
% Some rights reserved.
% CVSId : $Id: logica-booleana.tex 8 2014-02-04 21:01:21Z nicb $
%
\svnInfo $Id: logica-booleana.tex 8 2014-02-04 21:01:21Z nicb $

\setcounter{ms}{0}
\refstepcounter{ms}
\begin{frame}
    \frametitle{La logica booleana (\arabic{ms})}

	\begin{itemize}[<+- | alert@+->]

		\item Oltre all'aritmetica, che funziona come per tutti gli altri
		      sistemi di simboli, vengono spesso usate le seguenti
			  operazioni logiche (logica booleana):

			\begin{itemize}[<+- | alert@+->]

				\item {\bfseries AND}: $1010~``AND''~0011 = 0010$
					\hfill
					\begin{tabular}{c | c c}
						AND & 0 & 1\\
						\hline
						0 & 0 & 0\\
						1 & 0 & 1\\
					\end{tabular}

					{\itshape (se sono riposato {\bfseries E}
					fa bel tempo sono contento\ldots)}

				\end{itemize}

	\end{itemize}

\end{frame}

\refstepcounter{ms}
\begin{frame}
    \frametitle{La logica booleana (\arabic{ms})}

	\begin{itemize}[<+- | alert@+->]

		\item[~]

			\begin{itemize}[<+- | alert@+->]
			\setlength{\itemsep}{10mm}

				\item {\bfseries OR}: $1010~``OR''~0011 = 1011$
					\hfill
					\begin{tabular}{c | c c}
						OR & 0 & 1\\
						\hline
						0 & 0 & 1\\
						1 & 1 & 1\\
					\end{tabular}

					{\itshape (se sono riposato {\bfseries O}
					  fa bel tempo sono contento\ldots)}

				\item {\bfseries XOR}: $1010~``XOR''~0011 = 1001$
					\hfill
					\begin{tabular}{c | c c}
						XOR & 0 & 1\\
						\hline
						0 & 0 & 1\\
						1 & 1 & 0\\
					\end{tabular}

					{\itshape (sono contento se sono riposato {\bfseries E}
					 non fa bel tempo {\bfseries OPPURE} se non sono riposato
					 {\bfseries E} fa bel tempo\ldots)}

				\end{itemize}

	\end{itemize}

\end{frame}
