%
% Author: Nicola Bernardini <nicb@sme-ccppd.org>
%
% Copyright (c) 2004 Nicola Bernardini
% Copyright (c) 2004 Conservatorio "C.Pollini", Padova
%
% This work is licensed under the Creative Commons 
% Attribution-ShareAlike License. To 
% view a copy of this license, visit 
% http://creativecommons.org/licenses/by-sa/2.0/ 
% or send a letter to Creative Commons, 
% 559 Nathan Abbott Way, Stanford, California 94305, USA.
%
% Some rights reserved.
% CVSId : $Id: campionamento.tex 8 2014-02-04 21:01:21Z nicb $
%
\svnInfo $Id: campionamento.tex 8 2014-02-04 21:01:21Z nicb $

\setcounter{ms}{1}
\begin{frame}
    \frametitle{Frequenza di campionamento (\arabic{ms})}

	\begin{itemize}[<+- | alert@+->]

      \item Ogni quanto tempo devo misurare una funzione continua?

      \item Il \emph{teorema di Shannon} dice che
          
        \vspace{2ex}
       \begin{quote}
       \uncover<+- | alert@+->{una funzione periodica di frequenza data
          \`e completamente rappresentata da una serie di punti equi--spaziati nel tempo}
       \uncover<+- | alert@+->{se la frequenza di questi punti \`e almeno \emph{il doppio} della
       funzione periodica che si intende rappresentare.}
      \end{quote}

	\end{itemize}

\end{frame}

\refstepcounter{ms}
\begin{frame}
    \frametitle{Frequenza di campionamento (\arabic{ms})}

		\begin{center}
    \begin{figure}
			\pgfimage<+->[height=0.7\textheight]{\imagedir/wave-downsamp-1}
      \caption{\scriptsize Freq. fond: $300 Hz$, Periodo di campionamento: $250 {\mu}sec$ ($= 4000 Hz$)}
    \end{figure}
		\end{center}

\end{frame}


\refstepcounter{ms}
\begin{frame}
    \frametitle{Frequenza di campionamento (\arabic{ms})}

		\begin{center}
    \begin{figure}
			\pgfimage<+->[height=0.7\textheight]{\imagedir/wave-downsamp-2}
      \caption{\scriptsize Freq. fond: $300 Hz$, Periodo di campionamento: $500 {\mu}sec$ ($= 2000 Hz$)}
    \end{figure}
		\end{center}

\end{frame}


\refstepcounter{ms}
\begin{frame}
    \frametitle{Frequenza di campionamento (\arabic{ms})}

		\begin{center}
    \begin{figure}
			\pgfimage<+->[height=0.7\textheight]{\imagedir/wave-downsamp-3}
      \caption{\scriptsize Freq. fond: $300 Hz$, Periodo di campionamento: $2.5 msec$ ($= 400 Hz$)}
    \end{figure}
		\end{center}

\end{frame}



\refstepcounter{ms}
\begin{frame}
    \frametitle{Frequenza di campionamento (\arabic{ms})}

		\begin{center}
			\pgfimage<+->[height=0.7\textheight]{\imagedir/vari_samp}
		\end{center}

\end{frame}


\refstepcounter{ms}
\begin{frame}
    \frametitle{Frequenza di campionamento (\arabic{ms})}

	\begin{itemize}

		\item Quando la frequenza di campionamento non \`e sufficiente,
              si verifica il fenomeno del \emph{aliasing} (anche detto \emph{foldover})
			  (ripiegamento delle frequenze attorno alla met\`a
			  della frequenza di campionamento).

	\end{itemize}

\end{frame}

\refstepcounter{ms}
\begin{frame}
    \frametitle{Frequenza di campionamento (\arabic{ms})}

  \vspace{-1ex}
	\begin{itemize}

		\item \listento{run: audacity \exampledir/foldover.wav}{Esperimento:}

      \vspace{-1ex}
			\begin{itemize}

				\item un glissando (prima sinusoidale, poi complesso)
					  va da 4500 Hz a 6500 Hz;
					  su un canale \`e campionato a 44100 Hz,
                      sull'altro a 11025 Hz

					  \begin{center}
						\pgfimage[height=0.5\textheight]{\imagedir/foldover-explained}
					  \end{center}

			\end{itemize}

	\end{itemize}

\end{frame}

\refstepcounter{ms}
\begin{frame}
    \frametitle{Frequenza di campionamento (\arabic{ms})}

	\begin{itemize}

		\item Dato che l'orecchio pu\`o percepire
			  frequenze sino a 20000 Hz ca.,
			  \`e opportuno campionare a frequenze superiori a 40000 Hz.

	\end{itemize}

\end{frame}

\refstepcounter{ms}
\begin{frame}
    \frametitle{Frequenza di campionamento (\arabic{ms})}

	\begin{itemize}

		\item \listento{run: audacity \exampledir/downsample.wav}{Esperimento:}

			\begin{itemize}

				\item un brano campionato alle frequenze:
                      1 Hz, 10 Hz, 100 Hz, 980 Hz, 
                      5512.5 Hz, 11025 Hz, 22050 Hz, 44100 Hz

					  \begin{center}
						\pgfimage[height=0.58\textheight]{\imagedir/downsampling-explained}
					  \end{center}
			\end{itemize}

	\end{itemize}

\end{frame}
